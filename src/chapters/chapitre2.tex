%----------------------------------------------------------------------------------------
%	CHAPITRE 2
%----------------------------------------------------------------------------------------

\chapter{Test et stratégies de recherche}

\label{Chapter2} % For referencing the chapter elsewhere, use \ref{Chapter2} 

%----------------------------------------------------------------------------------------
%	SECTION 1
%----------------------------------------------------------------------------------------

\section{Le test logiciel (c'est quoi ? pourquoi ?)}

%----------------------------------------------------------------------------------------
%	SECTION 2
%----------------------------------------------------------------------------------------

\section{L'exécution symbolique ?}
%TODO 
Méthode pour simuler l'exécution d'un programme.
Elle collecte les contraintes des branches de décision et remplace les valeurs des données en mémoire par des valeurs symboliques à base de formules mathématiques.

%----------------------------------------------------------------------------------------
%	SECTION 3
%----------------------------------------------------------------------------------------

\section{État de l'art, tests et heuristiques de recherche}

\subsection{Article 1}
%TODO
Path exploration based  on Monte Carlo Tree Search for symbolic execution

Veut répondre à la problématique de l'explosion des chemins lors du parcours du graphe d'exécution symbolique d'un programme.
L'article compare les différents algorithme en fonction du nombre de blocs d'instruction parcouru.

\subsection{Article 2}
%TODO
Monte Carlo Tree Search for program synthesis
%TODO C'est quoi la synthèse de programme.
L'objectif de la synthèse de programme est de produire de manière automatique un exécutable d'un segment de code d'un programme qui garantis certains critères.
Méthode la plus utilisé auparavent: genetic programming.