%%%%%%%%%%%%%%%%%%%%%%%%%%%%%%%%%%%%%%%%%
% Masters/Doctoral Thesis 
% LaTeX Template
% Version 2.5 (27/8/17)
%
% This template was downloaded from:
% http://www.LaTeXTemplates.com
%
% Version 2.x major modifications by:
% Vel (vel@latextemplates.com)
%
% This template is based on a template by:
% Steve Gunn (http://users.ecs.soton.ac.uk/srg/softwaretools/document/templates/)
% Sunil Patel (http://www.sunilpatel.co.uk/thesis-template/)
%
% Template license:
% CC BY-NC-SA 3.0 (http://creativecommons.org/licenses/by-nc-sa/3.0/)
%
%%%%%%%%%%%%%%%%%%%%%%%%%%%%%%%%%%%%%%%%%

%----------------------------------------------------------------------------------------
%	PACKAGES AND OTHER DOCUMENT CONFIGURATIONS
%----------------------------------------------------------------------------------------

\documentclass[
12pt, % The default document font size, options: 10pt, 11pt, 12pt
%oneside, % Two side (alternating margins) for binding by default, uncomment to switch to one side a activer si on veut qu'il n'y ait pas de décalage de la marge une page sur deux et que la pagination soit une fois à droite et une fois à gauche (c'est le cas où l'on imprime recto-verso).
french, % ngerman for German
singlespacing, % Single line spacing, alternatives: onehalfspacing or doublespacing
%draft, % Uncomment to enable draft mode (no pictures, no links, overfull hboxes indicated)
%nolistspacing, % If the document is onehalfspacing or doublespacing, uncomment this to set spacing in lists to single
%liststotoc, % Uncomment to add the list of figures/tables/etc to the table of contents
%toctotoc, % Uncomment to add the main table of contents to the table of contents
%parskip, % Uncomment to add space between paragraphs
%nohyperref, % Uncomment to not load the hyperref package
headsepline, % Uncomment to get a line under the header
%chapterinoneline, % Uncomment to place the chapter title next to the number on one line
%consistentlayout, % Uncomment to change the layout of the declaration, abstract and acknowledgements pages to match the default layout
]{MastersDoctoralThesis} % The class file specifying the document structure

% \usepackage[frenchb]{babel}
\usepackage[utf8]{inputenc} % Required for inputting international characters
\usepackage[T1]{fontenc} % Output font encoding for international characters
\usepackage{mathpazo} % Use the Palatino font by default
\usepackage[
backend=bibtex]{biblatex} % Use the bibtex backend with the authoryear citation style (which resembles APA)

\addbibresource{bibliography/bibliography.bib} % The filename of the bibliography

\usepackage[autostyle=true]{csquotes} % Required to generate language-dependent quotes in the bibliography

\usepackage{algorithm}
\usepackage{algpseudocode}
%----------------------------------------------------------------------------------------

% Define some commands to keep the formatting separated from the content 
\newcommand{\keyword}[1]{\textbf{#1}}
\newcommand{\tabhead}[1]{\textbf{#1}}
\newcommand{\code}[1]{\texttt{#1}}
\newcommand{\file}[1]{\texttt{\bfseries#1}}
\newcommand{\option}[1]{\texttt{\itshape#1}}

%----------------------------------------------------------------------------------------

%----------------------------------------------------------------------------------------
%	MARGIN SETTINGS
%----------------------------------------------------------------------------------------

\geometry{
	paper=a4paper, % Chrvisange to letterpaper for US letter
	inner=2.5cm, % Inner margin
	outer=3.8cm, % Outer margin
	bindingoffset=.5cm, % Binding offset
	top=1.5cm, % Top margin
	bottom=1.5cm, % Bottom margin
	%showframe, % Uncomment to show how the type block is set on the page
}

%----------------------------------------------------------------------------------------
%	THESIS INFORMATION
%----------------------------------------------------------------------------------------
\thesistitle{Recherche heuristique pour la détection de divergences de comportement entre programmes dans le cadre de l'apprentissage de la programmation} % Your thesis title, this is used in the title and abstract, print it elsewhere with \ttitle
\supervisor{Emmanuel \textsc{Hyon} } % Your supervisor's name, this is used in the title page, print it elsewhere with \supname
\examiner{} % Your examiner's name, this is not currently used anywhere in the template, print it elsewhere with \examname
\author{Valentin \textsc{Bouquet}} % Your name, this is used in the title page and abstract, print it elsewhere with \authorname
\addresses{valentin-bouquet@hotmail.fr} % Your address, this is not currently used anywhere in the template, print it elsewhere with \addressname

\keywords{Heuristique de recherche, Apprentissage, Test logiciel, Génie logiciel, Exécution symbolique, Automatisation} % Keywords for your thesis, this is not currently used anywhere in the template, print it elsewhere with \keywordnames
\university{\href{http://\\
www.parisnanterre.fr/}{Université Paris Nanterre}} % Your university's name and URL, this is used in the title page and abstract, print it elsewhere with \univname
\department{\href{https://www.parisnanterre.fr/organisation/departement-de-mathematiques-et-informatique-343280.kjsp}{Département de mathématiques et informatique}} % Your department's name and URL, this is used in the title page and abstract, print it elsewhere with \deptname

\AtBeginDocument{
\hypersetup{pdftitle=\ttitle} % Set the PDF's title to your title
\hypersetup{pdfauthor=\authorname} % Set the PDF's author to your name
\hypersetup{pdfkeywords=\keywordnames} % Set the PDF's keywords to your keywords
}

\begin{document}
% \let\cleardoublepage\clearpage % A activer pour supprimer les pages blanches entre chapitres, table des matières, remerciements, etc.

\frontmatter % Use roman page numbering style (i, ii, iii, iv...) for the pre-content pages

\pagestyle{plain} % Default to the plain heading style until the thesis style is called for the body content

%----------------------------------------------------------------------------------------
%	TITLE PAGE
%----------------------------------------------------------------------------------------

\begin{titlepage}
\begin{center}

\vspace*{.06\textheight}
{\scshape\LARGE \univname \par}\vspace{1.5cm} % University name
\textsc{\Large Mémoire de fin d'études\\[0.5cm] MASTER 2 MIAGE}\\[0.5cm] % Thesis type

\HRule \\[0.4cm] % Horizontal line
{\huge \bfseries \ttitle\par}\vspace{0.4cm} % Thesis title
\HRule \\[1.5cm] % Horizontal line

\begin{minipage}[t]{1\textwidth}
\begin{center}
\emph{\large Présenté et soutenu par:}\\[0.3cm]
\href{http://www.github.com/vbouquet}{\LARGE \authorname} % Author name - remove the \href bracket to remove the link
\end{center}
\end{minipage}\\[1cm]

{\large \today}\\[1cm] % Date

\begin{minipage}[t]{1\textwidth}
\begin{center}
\emph{Sous la direction de:} \href{https://www.lip6.fr/actualite/personnes-fiche.php?ident=P216}{Monsieur \supname  Maitre de conférences} % Supervisor name - remove the \href bracket to remove the link  \\
\end{center}
\end{minipage}\\[2cm]

\includegraphics[scale=0.75]{figures/logo-paris-nanterre.jpg} % University/department logo - uncomment to place it
 
\vfill
\end{center}
\end{titlepage}

%----------------------------------------------------------------------------------------
%	QUOTATION PAGE
%----------------------------------------------------------------------------------------

% \vspace*{0.2\textheight}

% \noindent\enquote{\itshape }\bigbreak

% \hfill Valentin Bouquet

%----------------------------------------------------------------------------------------
%	Résumé (abstract)
%----------------------------------------------------------------------------------------

\newenvironment{abstract}%
{\cleardoublepage\thispagestyle{empty}\null\vfill\begin{center}%
\bfseries\abstractname\end{center}}%
{\vfill\null}

\begin{abstract}
Dans le cadre de l'apprentissage de la programmation et de l'évaluation des programmes des apprenants nous sommes confrontés à la complexité de la génération automatiques des critères d'évaluation. Écrire des tests pour s'assurer de la conformité d'un programme avec un comportement souhaité demande du temps et une certaine expertise. A cause de ces contraintes, la qualité des tests peut être négligé et il préférable de disposer d'outils qui permettent de détecter automatiquement les programmes dont le comportement est divergent.
Dans ce cadre là, nous présentons les grandes techniques du test logiciel et plus spécifiquement de la génération automatique de jeux de tests ainsi qu'une présentation des heuristiques de recherches simples et avec renforcement dans le but de palier au limitations des méthodes existantes et plus spécifiquement à l'exécution symbolique. Enfin nous présentons une ébauche de méthode pour résoudre notre problématique avec les techniques présentées précédemment.\\

\textbf{Mots clés:} \keywordnames
\end{abstract}

%----------------------------------------------------------------------------------------
%	Remeciements
%----------------------------------------------------------------------------------------
\ohead*{} % Numéro de page en haut
\ofoot*{\pagemark} % Numéro de page en bas
\begin{acknowledgements}

%TODO
Cette année est l'aboutissement de 5 années d'études universitaires, je tiens donc à saluer tous les enseignants et enseignants-chercheurs que j'ai pu côtoyer durant toutes ces années ainsi que tous les étudiants, de la licence au master de la \textit{MIASH} et de la \textit{MIAGE} de l'Université de Paris-Nanterre.\\

Je pense aussi aux opportunités qui se sont présentées à moi afin de poursuivre mes études sereinement dans le supérieure. Il ne fait aucun doute que le climat universitaire français est généreux et que sans lui il m'aurait été impossible de continuer comme je l'ai fait. J'espère que les générations futures seront tout aussi à mène d'en bénéficier et de savourer cette chance. En espérant que les entraves à ce système s'essoufflent et laissent place à un avenir plus prometteur.\\

Je remercie Monsieur Emmanuel Hyon pour son encadrement, son écoute et pour sa bienveillance envers ses étudiants ainsi que pour ses conseils qu'il a su me prodiguer, y compris pour la rédaction de ce mémoire.\\

Je remercie chaleureusement Monsieur François Delbot et Monsieur Jean-François Pradat-Peyre pour tout le soutien qu'ils m'ont porté et de m'avoir permis de travailler dans d'excellentes conditions. Je les remercie vivement pour tout les conseils, avis et discussions partagés qui m'ont ouvert vers de nouvelles horizons.
Sans leurs soutiens, je n'aurais jamais une seconde pu imaginer poursuivre un travail de recherche. J'espère que ces récents espoirs seront trouver satisfaction.

\end{acknowledgements}


%----------------------------------------------------------------------------------------
%	LIST OF CONTENTS/FIGURES/TABLES PAGES
%----------------------------------------------------------------------------------------

\tableofcontents % Prints the main table of contents

\listoffigures % Prints the list of figures

% \listoftables % Prints the list of tables

%----------------------------------------------------------------------------------------
%	ABBREVIATIONS
%----------------------------------------------------------------------------------------

% \begin{abbreviations}{ll} % Include a list of abbreviations (a table of two columns)

% \textbf{LAH} & \textbf{L}ist \textbf{A}bbreviations \textbf{H}ere\\
% \textbf{WSF} & \textbf{W}hat (it) \textbf{S}tands \textbf{F}or\\

% \end{abbreviations}

%----------------------------------------------------------------------------------------
%	THESIS CONTENT - CHAPTERS
%----------------------------------------------------------------------------------------

\mainmatter % Begin numeric (1,2,3...) page numbering
\pagestyle{thesis} % Return the page headers back to the "thesis" style

% Include the chapters of the thesis as separate files from the Chapters folder
% Uncomment the lines as you write the chapters

\chapter*{Introduction}
\addchaptertocentry{Introduction}

Ces dernières années des méthodes d'intelligence artificielles ont été appliqués dans de nombreux domaines, c'est le cas par exemple de l'ordinateur X qui a battu le joueur Y au jeu de GO grâce à un algorithme reposant sur la famille des Monte Carlo Tree Search, méthode de recherche basé sur l'apprentissage.

Émergence de solutions pratiques car la recherche est disponible (30ans de recherche d'IA) et que l'aire de big data ont permis l'émergeance de cas d'applications concrets.

Intérêt: utiliser les méthodes de recherches arborescentes informées permettant de trouver des solutions à un problème donné à forte combinatoire et ceci en un temps raisonnable.

Le test et la vérification logiciel est un domaine très important pour X, Y et Z.
La vérification de logiciel et la génération de tests est souvent limité à cause de la complexité des programmes et des ressources limités des ordinateurs actuels.
%TODO Quels sont les motivations ?
%TODO Intérêt des tests logiciels et de la vérification pour améliorer la qualité logiciel
%TODO Intérêt des différentes heuristiques et des parcours
%TODO Présenter le plan + objectifs

%TODO A supprimer (pour voir les citations en fait)
A\cite{judea-pearl-heuristics}
B\cite{MCTS-methods-survey}
C\cite{MTCS-symbolic-execution-path-exploration}
D\cite{MTCS-program-synthesis}
E\cite{MTCS-symbolic-execution-less-path}
F\cite{testing-and-machine-learning}
G\cite{test-data-generation-water-drop}
H\cite{symbolic-execution-machine-learning}
I\cite{testsurvey}
J\cite{shadow}    
K\cite{ART}    
L\cite{EFSM}    

%----------------------------------------------------------------------------------------
%	CHAPITRE 1
%----------------------------------------------------------------------------------------
\chapter{Stratégies de recherche} % Nom du premier chapitre

\label{Chapter1} % For referencing the chapter elsewhere, use \ref{Chapter1} 



%----------------------------------------------------------------------------------------
%	SECTION 1
%----------------------------------------------------------------------------------------
\section{Heuristiques}

% SOUS-SECTION 1
\subsection{Définitions et intérêt}
Les heuristiques sont des critères, méthodes ou principes utilisées pour sélectionner une solution efficace parmi un ensemble possible afin d'atteindre un ou plusieurs objectifs fixés\cite{judea-pearl-heuristics}.
Elles ne sont d'ailleurs pas toujours justes ou fiables dans toutes les situations et peuvent donc être hasardeuses.

Pour une grande majorité des problèmes complexes, déterminer une solution exacte nécessite d'évaluer un immense ensemble de choix. Le temps requis pour trouver cette solution peut être trop important et il est nécessaire parfois de faire des compromis pour obtenir une solution efficace en un temps raisonnable en utilisant une heuristique. 

Elles sont particulièrement utilisées pour répondre aux problèmes dits NP-complet. Ce sont les problèmes pour lesquels tous les algorithmes connus requièrent un temps exponentiel en pire cas pour être résolus. 
Un des exemples les plus connus et fréquemment enseigné en cours d'algorithmique est le problème du voyageur de commerce.

\subsubsection{Problème du voyageur de commerce (TSP)}
Soit un ensemble de \textit{n} villes réparties et un voyageur souhaitant toutes les parcourir une et une seule fois puis retourner à la ville d'où il est parti et ce en parcourant au total la distance la plus petite possible.

\begin{center}
    \includegraphics[scale=0.6]{../ressources/images/probleme_voyageur.png}
    \captionof{figure}{Un chemin possible pour résoudre le problème du voyageur de commerce.}
\end{center}

Il existe au total $n!$ chemins possibles soit dans notre cas 120. La ville de départ n'ayant aucune influence sur la longueur totale parcourue on peut donc réduire l'ensemble à $(n-1)!$ soit 24 chemins. Enfin chaque chemin pouvant être parcouru dans les deux sens sans impacté la distance,  on peut donc réduire l'ensemble final à $\frac{1}{2}(n-1)!$ soit 12 chemins.

% Table montrant l'évolution du nombre de chemins en fonction du nombre de villes
\begin{table}[h]
\centering
\begin{tabular}{ l|r }
  Villes & Chemins (solutions) \\
  \hline
  5 & 12 \\
  10 & 181400 \\
  15 & 43 589 145 600 \\
  20 & 1 216 451 004 088 320 000\\
\end{tabular}
  \caption{Évolution de l'ensemble des chemins en fonction du nombre de villes pour la résolution du problème du voyageur de commerce}
\end{table}

Le tableau ci-dessus rend compte de la rapidité à laquelle le nombre de chemins à évaluer grandit en fonction du nombre de villes. On parle d'\textit{explosion combinatoire}: c'est le fait qu'un problème se complexifie grandement lorsque le nombre de données à considérer augmente légèrement et peut rendre sa solution incalculable en un temps restreint (longévité humaine par exemple).

\subsubsection{Un algorithme glouton}
Pour trouver une solution (pas forcément la meilleure) au problème du voyageur de commerce, nous pouvons utiliser une heuristique simple en utilisant un algorithme glouton. Un tel algorithme repose sur le fait de dérouler les données de manière itérative en sélectionnant à chaque étape un optimum local. Ceci a pour effet de grandement diminuer le nombre de données à considérer et donc de répondre en partie à l'explosion combinatoire.

\begin{algorithm}
\caption{Problème du voyageur - un algorithme glouton} 
\label{algo-glouton-TSP}
\begin{algorithmic}
\Function{determiner\_chemin}{$villes, n$}
    \State $P := LISTE\_VIDE$
    \State \textbf{choisir un sommet} u \textbf{dans} villes
    \State $P := P \cup u$
    \While{$|P| \ne n$}
        \State $d := +\infty$
        \For{v \textbf{in} villes}
            \Comment{Évaluation de la ville la plus proche}
            \If{$distance(u, v) < d$}  
                \State $d := distance(c, v)$
                \State $u` := v$
            \EndIf
        \EndFor
        \State $u := u'$
        \State $P := P \cup u$
        \State $villes := villes - u$
    \EndWhile
    \State \Return $P$
\EndFunction
\end{algorithmic}
\end{algorithm}

L'algorithme \ref{algo-glouton-TSP} présente une heuristique simple pour répondre aux problème d'explosion des chemins en diminuant le nombre de données à évaluer. 
Depuis la ville de départ $u$ (choisit aléatoirement par exemple), il s'agit de sélectionner la ville la plus proche parmi les $n-1$ villes restantes. Puis de manière itérative, nous sélectionnons la prochaine ville la plus proche depuis la dernière ville sélectionnée et ceci jusqu'à ce que toutes les villes soient sélectionnées. 
A la première itération nous avons donc $n-1$ distance à évaluer puis nous en aurons $n-2$ à la deuxième. Au final cet algorithme doit évaluer $\frac{n(n-1)}{2}$ distances. 

% Table montrant l'évolution du nombre de chemins en fonction du nombre de villes
\begin{table}[h]
\centering
\label{table-comparaison-chemins}
\begin{tabular}{ l|r|r }
  Villes & Chemins & Chemins (algorithme glouton) \\
  \hline
  5  &  12                       & 10  \\
  10 & 181400                    & 45  \\
  15 & 43 589 145 600            & 105 \\
  20 & 1 216 451 004 088 320 000 & 190 \\
\end{tabular}
  \caption{Comparaison du nombre de chemins évalués pour la résolution du problème du voyageur de commerce avec un algorithme glouton (heuristique).}
\end{table}

Ceci montre un exemple simple d'utilisation d'une heuristique qui repose sur la découpe du problème en sous-problèmes pour réduire l'ensemble des données du domaine. L'inconvénient est qu'une telle méthode ne donne pas de garantie de résultât car le chemin le plus court possible n'est retourné que dans le meilleure des cas.
La sélection d'une heuristique pour répondre à un problème réside dans le compromis entre le temps requis pour obtenir une solution et la qualité de la solution retournée c'est à dire sa proximité avec la meilleure solution possible. 
Ces deux critères peuvent être évalués en moyenne, dans le pire cas possible, dans le meilleure cas possible ou bien les trois à la fois. \\

Les heuristiques sont intéressantes en dehors du domaine théorique car la majorité des problèmes pratiques ne nécessite pas d'établir la solution la plus optimale. On préférera trouver un équilibre entre la qualité de la solution obtenue et le coût pour trouver une telle solution qui est un critère non négligeable si l'on prend compte du contexte économique. On parle alors de problème de \textit{semi-optimisation} et plus particulièrement d'optimisation proche lorsque qu'il s'agit de trouver une solution dans un intervalle de coût définie ou de problème d'optimisation approximatif lorsqu'il s'agit de se rapprocher de l'optimum avec une probabilité importante.

%TODO Paragraphe de transition

% SOUS-SECTION 2
\subsection{Modélisation}
Pour établir une bonne heuristique et évaluer sa capacité à produire des solutions en un temps défini sur un problème donné, il convient de correctement le représenter.
De nombreux problèmes peuvent être formulés comme problème de satisfaction de contraintes - où l'on cherche des états ou des objets satisfaisant un certain nombre de critères - et d'optimisation de tâches. De plus, une heuristique doit pouvoir être automatisé et donc être résolu à l'aide des machines actuels.

Puisque toutes les recherches de solutions à un problème peuvent se résumer à la tâche de construire un object avec les caractéristiques données, les besoins\cite{judea-pearl-heuristics} pour la résolution avec un ordinateur sont les suivants:

\begin{enumerate}
\item Une structure de symbole appelée code ou base de données représentant les sous-ensembles des solutions potentielles.
\item Un ensemble d'opérations ou des règles de production qui modifient les symboles de la base de données pour produire un sous-ensemble de solutions plus fins ou précis.
\item Une procédure de recherche ou stratégie de contrôle qui décide quelles opérations sont à appliquer sur la base de données.
\end{enumerate}

\subsubsection{Définitions générales}
Les différentes façons de représenter nos problèmes repose majoritairement sur des modèles de graphe. Vous pouvez passer à la section suivante si vous possédez déjà des connaissances de base de théorie des graphes sinon nous décrivons brièvement les notions importantes ici:\\

{\setlength{\parindent}{0cm}\textbf{Graphe:}}

Un graphe est composée d'un ensemble de \textbf{nœuds} ou \textbf{sommets} reliés par des \textbf{arcs} ou \textbf{arêtes} pouvant être associées à des valeurs (par exemple la distance entre deux sommets) ou bien être dirigé (donnant la direction, on va d'un nœud à l'autre).
Dans notre cas, nos graphes auront toujours un nœud de départ appelé \textbf{nœud racine}.
L'ensemble des nœuds est le plus souvent noté $V$ et on note $E$ pour l'ensemble des arêtes du graphe. 
Un graphe est mathématiquement représenté de cette façon: $G = (V, E)$.
Le \textbf{degré} d'un sommet est le nombre d'arêtes de celui-ci.\\

\begin{center}
    \includegraphics[scale=0.6]{../ressources/images/example_graph.png}
    \captionof{figure}{Représentation d'un graphe à 7 sommets et 7 arêtes avec un chemin dessiné en bleu entre le sommet A et le sommet H {A, D, F,G, H}.}
\end{center}

{\setlength{\parindent}{0cm}\textbf{Arbre:}}

Un arbre est un graphe non orienté dans lequel chaque nœud (sauf le nœud racine) n'a qu'un seul parent. 
On désigne comme \textbf{feuille} un nœud n'ayant aucun fils.
Dans un arbre on définit la \textbf{hauteur} comme étant la longueur du chemin de la racine vers le nœud feuille le plus éloigné. On parle de \textbf{profondeur} quand il s'agit de la distance entre n'importe quel nœud feuille et le nœud racine.
On parle d'\textbf{arbre uniforme} pour désigner un arbre fini de hauteur $n$ dont tous les nœuds qui sont inférieur en profondeur à $n$ ont le même degré et où tous les nœuds de profondeur $n$ sont des feuilles.

\begin{center}
    \includegraphics[scale=0.6]{../ressources/images/example_tree.png}
    \captionof{figure}{Un arbre uniforme de profondeur 2 à 7 sommets et 6 arêtes}
\end{center}

%TODO (P.34) Regarder si c'est utile, peut être le déplacer. 
%{\setlength{\parindent}{0cm}\textbf{Recherche dans un graphe:}}
%Dans notre cas, nous recherchons des heuristiques pour répondre à l'explosion combinatoire et donc dans le cas d'une représentation sous forme de graphe nous pouvons aussi parler d'explosion des chemins. L'ensemble des données est parfois tellement importante qu'il est impossible de modéliser l'ensemble.
%Pour combler ces lacunes, il est possible d'utiliser des techniques de génération du graphe pas à pas.

% Beaucoup de problèmes sont décrits comme des tâches pour chercher des propriétés sur un graphe.
% Objectif: Trouver des méthodes efficaces pour trouver la solution rapidement dans le graphe.

Le choix d'une représentation pour encadrer un problème se fait en fonction des contraintes et des données mais ce choix n'est pas unique et peut être différent en fonction de l'approche souhaité.

\subsubsection{ET-OU graphe}
Le graphe \textit{ET-OU} (ou graphe de réduction de problème) est une modélisation destinée à représenter un problème comme étant la conjonction de plusieurs sous-problèmes qui peuvent être résolus indépendamment.
Cette représentation est principalement utilisée lorsqu'il s'agit de trouver une stratégie de recherche efficace, c'est par exemple le cas si l'on souhaite résoudre le problème de la pièce contrefaite:

Nous avons douze pièces de monnaie et parmi elles se trouve une pièce contrefaite, c'est à dire qui est soit plus légère ou plus lourde que les autres. L'objectif est de déterminer une stratégie pour identifier en au plus trois pesées (avec une balance) quelle est la pièce contrefaite. Il s'agit donc de sélectionner une suite d'actions de ce qui doit être pesé en premier pour avoir une chance d'identifier la pièce contrefaite.
Bien entendu, ce problème peut être résolu en énumérant la totalité des solutions possibles mais l'objectif ici est d'utiliser une heuristique pour identifier une stratégie qui permette de résoudre ce problème en un minimum de pesée (action).

Pour résoudre ce problème, il faut décider du nombre de pièces à comparer à chaque pesée. On peut par exemple décider de peser les pièces une à une, deux à deux, trois à trois, et ainsi de suite. Intuitivement, on sait que si à la première action l'on compare un sous-ensemble de pièces restreint en prenant seulement deux pièces, l'approche sera plus hasardeuse puisqu'à la prochaine action, le sous-ensemble restant risque d'être trop important pour identifier une pièce contrefaite. 
Par contre, si nous pesons deux pièces au hasard en premier, il est possible d'obtenir le résultât en une seule pesée si l'une des deux est soit plus légère ou plus lourde.

Dans ce problème, nous ne sommes pas que confronté au choix de la prochaine action à entreprendre mais aussi aux conséquences de celle-ci qui affecteront inévitablement les prochaines décisions et délimiterons le prochain sous-ensemble.

Pour cela nous utilisons donc le graphe de réduction de problème où les nœuds représentent les sous problèmes et les arcs les conséquences de l'action entreprise sur ce sous-problème. L'avantage de ce type de modélisation est qu'il permet de découper le problème initial en sous-problèmes indépendants grâce à une technique appelée « diviser pour régner ».\\

{\setlength{\parindent}{0cm}\textbf{Arêtes ET:}}

Mène à des sous-problèmes indépendants qui devront tous être résolus pour résoudre le problème associé au nœud père. Cet arc représente les changements dans la situations du problème.

{\setlength{\parindent}{0cm}\textbf{Arêtes OU:}}

Mène à des sous-problèmes alternatifs, dont l'un devra être résolu pour résoudre le problème associé au nœud père. Cet arc représente les différentes réactions possible après un tel changement.

\begin{center}
    \includegraphics[scale=0.8]{../ressources/images/ET_OU_arcs.png}
    \captionof{figure}{Respectivement les arcs ET et OU du graphe de réduction de problème.}
\end{center}

Nous pouvons donc modéliser notre problème sous forme de graphe où chaque décision prise depuis le nœud racine forme une solution possible résultant de la première action entreprise. Une solution n'est pas donc qu'un chemin dans le graphe, mais un sous-graphe de notre modèle commençant au nœud racine. La figure 1.5 donne un exemple d'une solution possible où l'action de comparer 2 pièces de monnaies est prise en premier.

\begin{center}
    \includegraphics[scale=0.8]{../ressources/images/counterfeit_problem_and_or_graph.png}
    \captionof{figure}{Problème de la pièce contrefaite représenté avec un ET-OU graphe\cite{judea-pearl-heuristics}}
\end{center}

%TODO Conclure + transition

\subsubsection{Représentation d'état}
Une représentation d'état consiste essentiellement en un ensemble de nœuds représentant chacun les états possibles du problème. Les arêtes entre les nœuds représentent les actions possibles d'un état à un autre. 
Chaque représentation d'état prend la forme d'un graphe ou d'un arbre.

Une représentation sous forme d'espaces états sera plutôt utilisé pour modéliser un problème de satisfaction de contraintes ou de recherche de chemin.
Si la solution peut être exprimée comme une séquence d'actions inconditionnelles ou comme un seul objet avec un ensemble de caractéristiques nous avons un problème de plus court chemin ou de satisfaction de contraintes qui est donc modélisable comme ceci.

Avant de représenter notre problème, il faut préalablement définir un ensemble de facteurs:
\begin{itemize}
\item Quel est l'objectif à atteindre ?
\item Quels sont les actions possibles ?
\item Quels informations doivent être représentées dans la description des états ?
\end{itemize}

Par exemple on peut souhaiter rechercher des erreurs dans un programme. Pour cela, nous pouvons modéliser chaque nœud comme étant un état du programme issu des différentes conditions de branchement où les valeurs concrètes en mémoire seraient remplacées par des valeurs symboliques. Il s'agit alors de parcourir le graphe, pour identifier d'éventuelles valeurs pour lesquelles le programme n'aurait pas le comportement souhaité.

\begin{center}
    \includegraphics[scale=0.6]{../ressources/images/state_space_graph.png}
    \captionof{figure}{Représentation sous forme d'état de l'exécution symbolique d'un programme}
\end{center}

%TODO (P.34) Définir toutes les notions pour l'explorations des noeuds
%Noeud étendu: Génération de tous les successeurs d'un noeud parant (mais pas les successeurs des successeurs je crois).
%Noeud exploré: Noeud visité
%Noeud généré: Génération d'un noeud successeur à partir de sont parant.


%----------------------------------------------------------------------------------------
%	SECTION 2
%----------------------------------------------------------------------------------------
\section{Procédures de recherches simples}

%TODO Petite transition introductif

%TODO Améliorer définition Procédure de recherche ou stratégie
Stratégie (politique): procédure de recherche déterminant l'ordre dans lequel les nœuds du graphe seront parcourus afin d'obtenir la solution souhaitée (ou une solution proche).
2 grands types de stratégies que sont: stratégie à l'aveugle, stratégie guidé.

%TODO 
\textbf{Recherche systématique non informé:}

On dit que la recherche n'est pas informée quand dans un graphe, la location de l'objectif n'altère pas l'ordre dans lequel les noeuds seront parcourus.
Ces stratégies sont souvent inefficaces et peu pratique dans le cas de larges problèmes.

\subsection{Recherche non informée}

\subsubsection{Parcours en largeur}
%TODO
Assigne une priorité plus importante pour les noeuds des premiers niveaux. Cette stratégie garantie de trouver une solution (la meilleure). Très couteuse par contre (on parcourt tout).

Variation: procédure du coût uniforme, variation du parcours en largeur. Elle ressemble beaucoup à l'algorithme de Dijkstra. les noeuds sont définis comme étant le coût total du chemin pour aller jusqu'à lui et il s'agit de parcourir à chaque fois les descendants du noeud avec la plus petite valeur (si problème minimisation).

\subsubsection{Parcours en profondeur}
%TODO
La priorité est donné aux noeuds les plus profonds du graphe.
Variations: Sous une à deux conditions définis on permet d'arrêter le parcours et de revenir à un autre noeud. Par exemple il est possible de reconnaître un noeud comme n'ayant jamais de fin ou comme une impasse ou n'ayant pas une propriété satisfaisante.

\subsubsection{Backtracking}
%TODO
C'est une version du parcours en profondeur qui applique la politique de dernier-entrée-premier-sortie (LIFO) pour la génération des nœuds au lieu de l'expansion.
A chaque nœud, seulement un successeur est généré et parcouru sauf si il ne remplit pas un critère donné. Dans le cas où il ne serait pas parcouru on revient à l'ancestre le plus proche parcouru avec au moins 1 nœud non généré.

Intérêt: Problème d'optmisation et de semi-optimisation. Si l'objectif est de trouver le coût minimal alors cette stratégie fonctionne comme nous pouvons parcourir le graphe tout en maintenant le coût minimal parcouru à tout instant t.

\subsubsection{Hill climbing}
%TODO P.35
La plus populaire. Depuis la position courante, sélectionner la montée la plus rapide.
Possibilité de trouver un chemin ne finissant jamais et sans solution.
Comme on ne peut pas passer 2 fois sur un noeud, il faut fermer un chemin si il n'est pas correct et on ne pourra plus le reprendre même pour optimiser son parcours. Il est donc possible de parcourir un chemin à partir de toutes les branches de profondeur 1 sans trouver de solution.

Avantages: 

- Quand on connait/possède des informations pour éviter les mauvais parcours et atteindre plus rapidement l'objectif fixé.

- Histoire sur les expansions commutatives: l'expansion d'un noeud ne compremet pas l'expansion des autres noeuds ou même de leurs descendants. Si une stratégie irrévocable est prise (mauvais chemin) alors cela n'affectera pas les prochains parcours.

\subsection{Recherche informée}

\subsubsection{Best-First}
%TODO
L'étape la plus évidente pour utiliser les informations de l'heuristique est de décider du prochain noeud à étendre d'abord (comme pour hill-climbing mais plus sophistiqué).
Il s'agit de prendre le meilleure noeud en le comparant à tous les noeuds déjà rencontrés.

%TODO Regarde algorithme (P.48)
Depuis le premier noeud on étend tous les successeurs et on prend le noeud n tel que f(n) soit le minimum (si min recherché). Puis on continue comme ça en comparant tous les noeuds ouverts non visités et en maintenant un état du chemin sur les successeurs.

\subsubsection{A*}
%TODO
C'est un algorithme dont l'objectif est de trouver une des meilleures solutions possibles. C'est donc une heuristique qui ne permet pas de toujours forcémment trouver la meilleure solution.

Il agit très similairement à l'algorithme de Dijkstra si ce n'est qu'il s'arête lorsque l'objectif est atteint (indépendemment du coût pour l'atteindre).

%\subsection{Stratégie à l'aveugle}
%TODO (P.34 paragraphe 3)
%L'ordre dans lequel les noeuds sont étendus dépend seulement des informations récoltées par la recherche mais n'est pas affecté par les informations des noeuds non explorés.

%\subsection{Stratégie guidée (informé)}
%TODO (P.34)
%Utilise des informations partiels du problème et de la nature de l'objectif pour aider à diriger vers la direction la plus prometteuse.

%\subsection{Stratégie informé: General Best first search AND/OR graph}
%TODO (P.52)
%Au lieu d'appliquer une seule fonction à chaque décision pour décider du prochain noeud à parcourir, nous appliquons deux fonctions.
%- f1: pour identifier la portion du sous-graphe la plus prometteuse.
%- f2: Pour identifier le noeud du sous-graphe sélectionné grâce à f1 et étant le plus prometteur.

%----------------------------------------------------------------------------------------
%	SECTION 3
%----------------------------------------------------------------------------------------

\section{Procédures de recherches avec apprentissage}

\subsection{Définitions}

\subsection{Monte Carlo Tree Search}
%TODO
Heuristique pour trouver des décisions optimales dans un arbre de décision.
%TODO Idéalement il faudrait trouver une bonne définition à référencer dans le mémoire.

Algorithme:
Un arbre est construit de manière incrémental et asynchrone. Pour chaque itération de l'algorithme, une politique est utilisée pour trouver le noeud le plus important de l'arbre actuel.
La politique d'arbre essaie d'équilibrer l'exploration de l'arbre (regarder dans les zones qui n'ont pas été essayées) et l'exploitation (regarder dans les zones qui ont l'air prometteuses).

Depuis un noeud une simulation est lancée et l'arbre se met à jour en fonction du résultât. Cela implique l'apparition d'un noeu noeud correspondant à l'action entreprise par l'algorithme.

Les mouvements sont effectués pendant la simulation selon une politique d'arbre pré-défini par défaut qui dans le plus simple des cas consiste à effectuer des mouvements aléatoires uniforme.

MCTS n'a besoin que de l'état terminal de la simulation précédente pour effectuer la suivante. Il n'utilise pas les états intermédiaires (je suppose qu'on parle d'état d'autres chemins/noeud qui auraient pu être évalués).
L'avantage est que cela réduit grandement les connaissances nécessaires à l'exécution de la méthode.

Même si l'lgorithme est efficace sur une grande varieté de problème, le vrai bénéfice d'utiliser MCTS est lorsqu'il est adapté au domaine du problème.

Méthode Monte-Carlo:
(Différent de MCTS)
Evaluation de la récompose: Q(s, a) = 1 / N(s, a) ... % (P.2)

\subsubsection{Définitions et intérêt}
%TODO

\subsubsection{Famille d'algorithmes}
%TODO
Deux concepts fondamentaux:

- La valeur réelle d'une action peut être approchée en utilisant une simulation aléatoire.

- Ces valeurs peuvent être utilisées pour ajuster la politique vers une stratégie du meilleur d'abord (Best-First).

L'algorithme construit progressivement un arbre de décision guidé par les résultâts des explorations précédentes.
L'arbre est utilisé pour estimer les valeurs associées à chaque mouvement.

Algorithme: 

Basique: Construction itérative d'un arbre de recherche jusqu'à ce qu'un critère prédéfini soit atteint. Souvent une limite de de calcul comme le temps d'exécution, ou la saturation de la mémoire.
%TODO (P.6 Fig 2. Schéma de l'algorithme, surement à mettre dans le mémoire.

1. Sélection: depuis le noeud racine, une politique est appliquée pour sélectionner les noeuds pour atteindre le noeud le plus important à étendre (noeud non visité et non terminal).

2. Expansion: Un ou plusieurs noeuds son ajoutés pour étendre l'arbre (choix en fonction des actions disponibles).

3. Simulation: une simulation est appliquée depuis le nouveau noeud en fonction d'une politique par défaut pour produire un résultât.

4. Backpropagation: Le résultât de la simulation est remotée à travers les noeuds sélectionnés pour arriver au noeud ajouté pour mettre à jour ses statistiques.

Deux politiques: 

- La politique de l'arbre: sélectionner ou créer un noeud feuille depuis les noeuds déjà parcourus/ajoutés.

- La politique par défaut: estimer la valeur d'un état non terminal (noeud ajouté) pour produire une estimation de sa valeur.

%TODO Variation UCTS

\subsubsection{Applications}

%TODO Montrer les cas d'applications les plus populaires comme pour le jeu de GO.

%TODO Montrer d'autres cas d'applications
Des applications pour des problèmes du plus court chemin (problème du voyageur). Il est efficace pour le problème du voyageur canadien où certain des chemins peuvent être bloqués avec une certaine probabilité (utilisation d'une variante UCT).

\subsection{Théorie de la décision}
%TODO
Elle combine les théories probabilistiques avec des théories utilitaires (heuristiques) pour offrir une approche formelle pour la prise de décision dans l'incertain.

Processus de décision Markovien: Modélise de manière séquentielle des problèmes de décision dans un environnement entièrement observable.
Les décisions sont modélisées comme un ensemble d'état, action dans lequel chacun des prochains états est évalués grâce à une distribution de probabilité en fonction de l'état courant et de l'action entreprise.
Une politique est une correspondance entre états et actions en spécifiant quel action doit être entreprise depuis chaque état.
L'objectif et de déterminer la politique qui permette de maximiser la récompose.
La fonction de transition évalue la probabilité depuis l'état s et l'action a de se retrouver dans l'état s' (les états sont incertains).


Processus de décision Markovien partiellement  observable (POMDP):
Contrairement à l'approche MDP, l'oracle n'a qu'une information partielle de l'état courant.

\subsection{Les autres procédures}

%----------------------------------------------------------------------------------------
%	CHAPITRE 2
%----------------------------------------------------------------------------------------
\chapter{Procédures de recherches}

\label{Chapter2} % For referencing the chapter elsewhere, use \ref{Chapter2} 

Nous avons définie la notion d'heuristique de recherche, présenté quelques problèmes types rencontrés en y appliquant des heuristiques simples et intuitives ainsi que des contraintes rencontrées lorsque les données à parcourir sont trop importantes et développé différentes méthodes pour modéliser nos problèmes. 
Ce chapitre présente quelques algorithmes et méthodes connus pour parcourir et rechercher des propriétés dans nos graphe (modèle).
Ces algorithmes définissent des stratégies ou politiques de recherche déterminant l'ordre dans lequel les nœuds du graphe seront parcourus afin d'obtenir la solution solution souhaitée ou une solution proche dans le cas d'heuristique.

Dans un premier temps nous présentons les procédures simples avec les algorithmes de parcours de graphe les plus populaires puis nous continuerons sur les recherches avec apprentissage.

%----------------------------------------------------------------------------------------
%	SECTION 1
%----------------------------------------------------------------------------------------
\section{Procédures de recherches simples}
%TODO Petite transition introductif
Nous présentons deux grands types de stratégies que sont les stratégies informées ou à l'aveugle et les stratégies guidées ou informées.

%\subsection{Hill climbing}
%TODO Doit-on faire ce type de parcours ? Il n'est ni informée ni non informée et il faut donc une nouvelle section (P.35)
%La plus populaire. Depuis la position courante, sélectionner la montée la plus rapide.
%Possibilité de trouver un chemin ne finissant jamais et sans solution.
%Comme on ne peut pas passer 2 fois sur un noeud, il faut fermer un chemin si il n'est pas correct et on ne pourra plus le reprendre même pour optimiser son parcours. Il est donc possible de parcourir un chemin à partir de toutes les branches de profondeur 1 sans trouver de solution.

%Avantages: 

%- Quand on connait/possède des informations pour éviter les mauvais parcours et atteindre plus rapidement l'objectif fixé.

%- Histoire sur les expansions commutatives: l'expansion d'un noeud ne compremet pas l'expansion des autres noeuds ou même de leurs descendants. Si une stratégie irrévocable est prise (mauvais chemin) alors cela n'affectera pas les prochains parcours.

\subsection{Recherche non informée}
On dit que la recherche n'est pas informée quand dans un graphe, la location de l'objectif n'altère pas l'ordre dans lequel les nœuds seront parcourus.
Ces stratégies sont souvent inefficaces et peu pratiques dans le cas de larges problèmes mais elles ont l'avantages d'être simples et peuvent être appliquées à tout problème puisqu'elles ne prennent  pas compte des données propres au domaine.

\subsubsection{Parcours en profondeur (DFS)}
Le parcours en profondeur donne la priorité aux nœuds les plus profonds du graphe et s'applique plus particulièrement au graphe de type arbre ou la profondeur est clairement définie.
Pour les autres types de graphe, il faudra définir la notion de profondeur et la direction dans laquelle se diriger.

Ce type de parcours assure de trouver la solution, puisqu'en pire cas toutes les solutions seront énumérées ce qui rend cet algorithme peu efficace dans de nombreux cas.\\
%TODO lesquels ?

%Variante backtracking
{\setlength{\parindent}{0cm}\textbf{backtracking:}}

Le \textit{backtracking} (ou retour sur trace) est une variante du parcours en profondeur qui permet de revenir en arrière à chaque action afin de minimiser le nombre de sommets du graphe à parcourir quand on se retrouve dans certaines situations qui ne correspondent pas à l'objectif recherché (propriétés d'un nœud non conforme).

%TODO
% C'est une version du parcours en profondeur qui applique la politique de dernier-entrée-premier-sortie (LIFO) pour la génération des nœuds au lieu de l'expansion.

A chaque nœud, seulement un successeur est parcouru sauf si il ne remplit pas un critère donné. Dans le cas où il ne serait pas parcouru on revient à l'ancêtre le plus proche parcouru possédant au moins 1 sommet fils qui n'a pas été encore parcouru.

\begin{center}
    \includegraphics[scale=0.6]{../ressources/images/depth_first_search.png}
    \captionof{figure}{Ordre dans lequel les sommets sont visités avec un parcours en profondeur (gauche) et sa variante (droite): le backtracking.}
\end{center}

Les problèmes d'optimisation ou de semi-optimisation tire parti de ce type de parcours. Si l'objectif est de trouver le coût minimal alors cette stratégie est efficace puisqu'elle permet de parcourir le graphe tout en maintenant le coût à tout instant t et ceci en priorisant les sommets les plus proches de la racine.

\subsubsection{Parcours en largeur (BFS)}
Contrairement au parcours en profondeur, le parcours en largeur donne la priorité au nœuds des premiers niveaux du graphe, c'est à dire aux sommets les plus proches de la racine qui seront parcourus en premier. Elle s'applique plus facilement aux arbres puisque la notion de niveau est naturellement défini, contrairement aux autres types de graphe ou il faudra définir ce qu'on entend par largeur pour orienter le sens du parcours.

Cette stratégie garantie aussi de trouver la meilleure solution possible puisque en pire cas, tout le graphe sera parcouru. C'est donc une méthode qui peut rapidement devenir très couteuse.\\

%Variante procédure du coût uniforme
{\setlength{\parindent}{0cm}\textbf{Procédure du coût uniforme:}}

La procédure du coût une uniforme est une variante du parcours en largeur. Au lieu de procéder au parcours des nœuds d'une même profondeur, le parcours se fait en fonction du coût des nœuds. Chaque nœud est exprimé comme étant le coût du chemin menant à lui depuis le nœud racine et la stratégie est accomplie en parcourant toujours le nœud avec le coût le plus faible.

\begin{center}
    \includegraphics[scale=0.5]{../ressources/images/breath_first_search.png}
    \captionof{figure}{Ordre dans lequel les sommets sont visités avec un parcours en largeur (gauche) et sa variante (droite): la procédure du coût uniforme.}
\end{center}

\subsection{Recherche informée}
Nous avons vu précédemment des stratégies qui ne prenaient pas en compte les données du problème autre que celles définies dans les sommets et les arêtes. Parfois il est possible de tirer parti d'informations qui ne sont pas dans le graphe pour diriger les recherches afin de parcourir les sommets qui ont l'air d'être les plus prometteurs.
Cette section présente l'intérêt de combiner des méthodes (fonctions) heuristiques aux parcours classiques.

\subsubsection{Best-First search (le meilleur d'abord)}
\textit{Best-first search} est un algorithme qui parcourt le graphe en étendant le nœud le plus prometteur d'abord. Contrairement aux heuristiques gloutonnes qui elles aussi sélectionnent les meilleurs éléments d'abord, \textit{best-first search} le fait par évaluation, c'est à dire en estimant la valeur d'un nœud avant de le parcourir à partir de données du problème qui ne sont pas présent dans le graphe, comme une fonction heuristique qui permet d'estimer la valeur d'un nœud (récompense). De plus l'objectif de cette méthode n'est pas seulement de sélectionner le sommet le plus prometteur parmi un ensemble de nœud à prochainement parcourir mais de le sélectionner en comparant tous les sommets déjà rencontrés.

La promesse, c'est à dire l'estimation de la qualité d'un nœud est estimé grâce à une fonction heuristique $f(n)$ qui peut dépendre des données du sommet $n$, de la description de l'objectif recherché, des informations récoltées lors du parcours de recherche et de toutes informations supplémentaires sur le domaine.\\

%TODO Définir fonction heuristique d'évaluation

%TODO Présenter l'algorithme

%TODO Montrer l'algorithme tel qu'il est présent dans le livre

\textbf{L'algorithme best-first décrit par Judea pearl\cite{judea-pearl-heuristics}}
\begin{enumerate}
\item Mettre le nœud racine (initial) dans la liste $NOEUDS OUVERTS$.
\item Si la liste $NOEUDS OUVERTS$ est vide, arrêter le programme: pas de solution.
\item Déplacer le nœud le plus prometteur de $NOEUDS OUVERTS$ c'est à dire celui pour lequel la fonction heuristique \textit{f} est minimum (ou maximum dans un problème de maximisation) dans la liste $NOEUDS FERMES$. On nomme ce nœud $n$.
\item Si l'un des successeurs de $n$ est l'objectif alors retourné le chemin vers ce nœud.
\item Pour chacun des successeurs $n'$ de n:
    \begin{enumerate}
    \item Calculer $f(n')$.
    \item Si $n'$ n'était ni dans la liste $NOEUDS OUVERTS$ ni $NOEUDS FERMES$, l'ajouter dans la liste $NOEUDS OUVERTS$. Attaché l'évaluation de l'heuristique au nœud ainsi qu'un lien vers le nœud $n$.
    \item Sinon comparer le résultât de la fonction heuristique avec le résultât enregistré (qui a été précédemment attaché au nœud). Si l'ancienne évaluation est meilleure (inférieure ou supérieure selon le problème de minimisation ou maximisation) alors passer à l'étape suivante. Si l'ancienne évaluation est moins bonne, remplacer l'évaluation ainsi que le lien  pour pointer vers le nœud $n$. Si le nœud $n$ est dans la liste $NOEUDS FERMES$, le déplacer dans $NOEUDS OUVERTS$.
    \end{enumerate}
\item Retourner à l'étape 2.
\end{enumerate}

%TODO Conclure sur l'intérêt de Best-first, pourquoi on l'utilise ? (P.49)
%TODO Définir clairement la différence avec la procédure du coût uniforme.

\subsubsection{Algorithmes best-first spécialisés}
Pour explorer un minimum de sommets possibles dans un graphe à la recherche d'un chemin optimal, un algorithme de recherche doit constamment faire les choix les plus informés possibles pour décider du prochain prochain nœud à explorer.

L'algorithme \textit{best-first search} n'est que la patron d'une stratégie à appliquer et nécessite d'être plus amplement définie pour être implémenté concrètement.
Il ne spécifie pas comment la fonction heuristique $f$ est calculé ni d'où les informations pour décider quel est le meilleur choix possible proviennent ou même comment elles se propagent sur le graphe pourtant se sont les éléments indispensables à une recherche efficace. \\

{\setlength{\parindent}{0cm}\textbf{A*:}}

L'algorithme \textit{A*} est un algorithme de type \textit{best first} et est aussi une extension de l'algorithme de Dijkstra\cite{dijkstra} un des plus populaires pour résoudre le problème du plus court chemin. 

L'algorithme A* définie\cite{description-a*} sa fonction d'évaluation comme suivant: 

\begin{center}
    $f(n) = g(n) + h(n)$
\end{center}

où $g(n)$ est le coût d'un chemin optimal du sommet racine $s$ vers le sommet $n$ et où $h(n)$ est le coût d'un chemin optimal du sommet $n$ vers le sommet $a$, un éventuel nœud objectif.\\    

%TODO Compléter les informations sur A* et donner un exemple où il est utile

\textbf{L'algorithme A*\cite{description-a*}}
\begin{enumerate}
\item Mettre le nœud racine (initial) dans la liste $NOEUDS OUVERTS$.
\item Si la liste $NOEUDS OUVERTS$ est vide, arrêter le programme: pas de solution.
\item Déplacer le nœud le plus prometteur de $NOEUDS OUVERTS$ c'est à dire celui pour lequel la fonction heuristique \textit{f} est minimum dans la liste $NOEUDS FERMES$. On nomme ce nœud $n$.
\item Si $n$ est l'objectif alors retourné le chemin vers ce nœud.
\item  Pour chacun des successeurs $n'$ de n:
    \begin{enumerate}
    \item Si $n'$ n'est pas dans $NOEUDS OUVERTS$ ou $NOEUDS FERMES$, estimé $h(n')$ (une estimation du coût du meilleur chemin de $n'$ vers un éventuel nœud objectif) et calculer $f(n') = g(n') + h(n')$ où $g(n') = g(n) + c(n, n')$ and $g(s) = 0$ où s est le nœud racine.
    \item Si $n'$ est dans $NOEUDS OUVERTS$ ou  $NOEUDS FERMES$. 
    \item Si $n'$ la condition de l'étape précédente est vraie et que $n'$ est dans la liste de $NOEUDS FERMES$ alors déplacer $n'$ dans $NOEUDS OUVERTS$.
    \end{enumerate}
\item Retourner à l'étape 2. 
\end{enumerate}

%TODO Conclusion + transition

%----------------------------------------------------------------------------------------
%	SECTION 3
%----------------------------------------------------------------------------------------

\section{Procédures de recherches avec apprentissage}


\subsection{Définitions}

\subsection{Monte Carlo Tree Search}
%TODO
Heuristique pour trouver des décisions optimales dans un arbre de décision.
%TODO Idéalement il faudrait trouver une bonne définition à référencer dans le mémoire.

%Algorithme:
Un arbre est construit de manière incrémental et asynchrone. Pour chaque itération de l'algorithme, une politique est utilisée pour trouver le noeud le plus important de l'arbre actuel.
La politique d'arbre essaie d'équilibrer entre l'exploration de l'arbre (regarder dans les zones qui n'ont pas été essayées) et l'exploitation (regarder dans les zones qui ont l'air prometteuses).

Depuis un nœud une simulation est lancée et l'arbre se met à jour en fonction du résultât. Cela implique l'apparition d'un nœud correspondant à l'action entreprise par l'algorithme.

Les mouvements sont effectués pendant la simulation selon une politique d'arbre pré-défini par défaut qui dans le plus simple des cas consiste à effectuer des mouvements aléatoires uniforme.

MCTS n'a besoin que de l'état terminal de la simulation précédente pour effectuer la suivante. Il n'utilise pas les états intermédiaires.
L'avantage est que cela réduit grandement les connaissances nécessaires à l'exécution de la méthode.

Même si l'algorithme est efficace sur une grande variété de problème, le vrai bénéfice d'utiliser MCTS est lorsqu'il est adapté au domaine du problème.

Méthode Monte-Carlo:
%(Différent de MCTS)
Évaluation de la récompense: 

$Q(s, a) = 1 / N(s, a)$ ... % (P.2)

\subsubsection{Définitions et intérêt}
%TODO

\subsubsection{Famille d'algorithmes}
%TODO
Deux concepts fondamentaux:

- La valeur réelle d'une action peut être approchée en utilisant une simulation aléatoire.

- Ces valeurs peuvent être utilisées pour ajuster la politique vers une stratégie du meilleur d'abord (Best-First).

L'algorithme construit progressivement un arbre de décision guidé par les résultâts des explorations précédentes.
L'arbre est utilisé pour estimer les valeurs associées à chaque mouvement.

Algorithme: 

Basique: Construction itérative d'un arbre de recherche jusqu'à ce qu'un critère prédéfini soit atteint. Souvent une limite de de calcul comme le temps d'exécution, ou la saturation de la mémoire.
%TODO (P.6 Fig 2. Schéma de l'algorithme, surement à mettre dans le mémoire.

1. Sélection: depuis le noeud racine, une politique est appliquée pour sélectionner les noeuds pour atteindre le noeud le plus important à étendre (noeud non visité et non terminal).

2. Expansion: Un ou plusieurs noeuds son ajoutés pour étendre l'arbre (choix en fonction des actions disponibles).

3. Simulation: une simulation est appliquée depuis le nouveau noeud en fonction d'une politique par défaut pour produire un résultât.

4. Backpropagation: Le résultât de la simulation est remotée à travers les noeuds sélectionnés pour arriver au noeud ajouté pour mettre à jour ses statistiques.

Deux politiques: 

- La politique de l'arbre: sélectionner ou créer un noeud feuille depuis les noeuds déjà parcourus/ajoutés.

- La politique par défaut: estimer la valeur d'un état non terminal (noeud ajouté) pour produire une estimation de sa valeur.

%TODO Variation UCTS

\subsubsection{Applications}

%TODO Montrer les cas d'applications les plus populaires comme pour le jeu de GO.

%TODO Montrer d'autres cas d'applications
%Des applications pour des problèmes du plus court chemin (problème du voyageur). Il est efficace pour le problème du voyageur canadien où certain des chemins peuvent être bloqués avec une certaine probabilité (utilisation d'une variante UCT).

%\subsection{Théorie de la décision}
%TODO
%Elle combine les théories probabilistiques avec des théories utilitaires (heuristiques) pour offrir une approche formelle pour la prise de décision dans l'incertain.

%Processus de décision Markovien: Modélise de manière séquentielle des problèmes de décision dans un environnement entièrement observable.
%Les décisions sont modélisées comme un ensemble d'état, action dans lequel chacun des prochains états est évalués grâce à une distribution de probabilité en fonction de l'état courant et de l'action entreprise.
%Une politique est une correspondance entre états et actions en spécifiant quel action doit être entreprise depuis chaque état.
%L'objectif et de déterminer la politique qui permette de maximiser la récompose.
%La fonction de transition évalue la probabilité depuis l'état s et l'action a de se retrouver dans l'état s' (les états sont incertains).


%Processus de décision Markovien partiellement  observable (POMDP):
%Contrairement à l'approche MDP, l'oracle n'a qu'une information partielle de l'état courant.

%\subsection{Les autres procédures}

%----------------------------------------------------------------------------------------
%	CHAPITRE 3
%----------------------------------------------------------------------------------------

\chapter{Application} % Nom du deuxième chapitre

\label{Chapter3} % For referencing the chapter elsewhere, use \ref{Chapter1} 

%----------------------------------------------------------------------------------------
%	SECTION 1
%----------------------------------------------------------------------------------------

\section{Présentation de mon cas}

%----------------------------------------------------------------------------------------
%	SECTION 2
%----------------------------------------------------------------------------------------

\section{Raisonnement}

%----------------------------------------------------------------------------------------
%	SECTION 3
%----------------------------------------------------------------------------------------

\section{Application}
%----------------------------------------------------------------------------------------
%	CHAPITRE 4
%----------------------------------------------------------------------------------------
\chapter{Application}

\label{Chapter2} % For referencing the chapter elsewhere, use \ref{Chapter2} 

%----------------------------------------------------------------------------------------
%	SECTION 1
%----------------------------------------------------------------------------------------
\section{Contexte}

\subsection{Apprentissage de la programmation et tests}
Maintenant que nous avons vu différentes techniques de génération automatiques de tests, leurs tenants et aboutissants ainsi que la notion d'heuristique et les algorithmes de recherche avec renforcement les plus populaires, rappelons nos objectifs.\\

Dans le cadre de l'apprentissage de la programmation, nous souhaitons évaluer des programmes sources d'apprenants. Pour cela nous avons fait le choix de proposer une approche comparative, l'évaluateur soumet le code de la correction et les spécifications sous forme de formules mathématiques ou de simples phrases en langue naturel que doivent suivre les apprenants pour réaliser un programme qui exécute le comportement souhaité.
Pour cela, nous mettons à disposition des outils qui facilitent la comparaison à l'exécution du programme soumis par l'apprenant au programme de la correction.
Bien que ces tests soient très pratiques, leur écriture est encore laborieuse puisqu'elle demande un investissement conséquent et une certaine expertise. De plus, les tests sont souvent négligés par les évaluateurs et de nombreux programmes qui ne devraient pas être corrects le sont par manque de couverture des différents cas d'exécution.
Nous souhaitons donc disposer d'outils pour générer des tests afin de détecter de manière automatisé les divergences de comportements entre deux programmes.

\subsection{Limitations des méthodes existantes}
%Problématique limitation
Nous avons vu que de nombreuses méthodes existent et font l'objet de recherche intensive dans le domaine du génie logiciel dans un objectif d'améliorer la qualité des programmes testés en facilitant la génération et en améliorant la qualité des jeux de tests générés. Nous avons aussi remarquer que ces méthodes sont dans certains cas assez limités et sont difficiles à réaliser dans des cas concrets.

C'est le cas de l'exécution symbolique qui est limitée face à l'explosion du nombre de chemins d'exécutions possibles d'un programme. Dans nos cas d'utilisations c'est une contrainte que nous pourrions rencontrer fréquemment puisqu'une simple fonction avec des boucles imbriquées et des conditions suffit à mettre à bout bon nombre d'outils sur une machine standard. De plus, les contraintes de chemins dépendant d'opérations non linéaires (sin, cos, multiplication, ...) contraignent aussi fortement la résolution des équations par les solveurs.

Des approches aléatoires sont aussi possibles mais sont souvent peu efficaces et plus le domaine d'entrée d'un programme est grand, moins les valeurs d'entrées générées auront une chance d'être pertinentes.

Les méthodes de \textit{Search-Based Testing} proposent de contourner les contraintes rencontrées grâce à l'utilisation de procédures de recherches pour optimiser la génération des tests en fonction d'un objectif précis. Malheureusement, cette objectif ce résume bien trop souvent à un objectif de couverture de branche qui reste très limité puisqu'il ne tient pas compte du comportement du programme en fonction des valeurs d'entrées.

\subsection{Objectifs}
%Ce que nous souhaitons faire
Comme nous disposons d'un cas très spécifique grâce à notre cadre d'apprentissage de la programmation, c'est à dire que nous avons une configuration avec plusieurs programmes ayant pour objectif de s'exécuter à l'identique du programme de la correction. Nous pensons qu'il serait judicieux d'utiliser une heuristique adaptée à notre cas en les complétant à des techniques de renforcement pour affiner et améliorer le parcours de chaque programme analysé.

%----------------------------------------------------------------------------------------
%	SECTION 2
%----------------------------------------------------------------------------------------
\section{Méthode proposée}

\subsection{Présentation}
Nous possédons le code source du programme de la correction qu'on notera $C$ et de $n$ code source provenant de $n$ apprenants ayant tentés de produire un programme suivant les consignes de l'exercice. Les programmes des apprenants sont note $Pi$ où i représente le numéro du programme parmi l'ensemble.

\subsubsection*{Étape de filtrage}
Nous définissons une première étape de test qui utilise $k$ valeurs d'entrées générées aléatoirement mais suffisamment éloignées les unes des autres. Ces jeux de tests serviront à filtrer les programmes dont le comportement s'éloigne trop de ce qui était attendu. 
Au final, nous souhaitons garder pour l'analyse uniquement les programmes qui se rapproche de la solution, où il serait pertinent de trouver un ou plusieurs jeux de tests appropriés pour découvrir une divergence de comportement. 
L'approche aléatoire avec un jeu de données disparates et suffisant nous permet de nous conforté dans l'idée que ces programmes ne correspondent pas au résultat attendu.

Cette étape de filtrage est importante, puisque nous souhaitons minimiser la charge de calcul lors des prochaines analyses. De plus, comme nous souhaitons classifier les instructions des programmes des apprenants, il est important que ces instructions soient utiles à l'exécution pour effectuer le comportement souhaité. Au moins, en retirant les programmes qui sont faux, nous retirons les potentiels programmes n'ayant aucun rapport avec l'exercice, ce qui permet de ne pas polluer la classification. 

Toujours dans l'optique de filtrage, nous pourrions appliquer des techniques de \textit{slicing} qui permettent de ne garder que le code nécessaire à une certaine utilisation mais ces techniques sont difficilement applicables à des programmes courts, ce qui représente la majorité des cas dans notre cadre. 
De plus, il faut noter qu'au préalable à l'analyse de comportement d'un programme, nous menons une analyse structurelles sur les programmes pour vérifier que certaines instructions sont bien présentes. Ces tests peuvent être éliminatoires et permettent de filtrer plus amplement les programmes.

\subsection*{Démarche d'analyse}
Nous prenons un programme $Pi$ au hasard parmi ceux ayant passé la première étape de filtrage et on applique notre analyse. Celle ci est un mixte entre l'exécution symbolique et l'heuristique du Monte Carlo Tree Search où nous utiliserons en fait l'exécution concolique lors de la simulation.

Notre premier programme est donc analysé symboliquement de la façon suivante:

\subsubsection*{Sélection}
Nous allons faire des choix, dans un premier temps aléatoire, pour décider à chaque nœud (représentant la contrainte séparant des chemins d'exécutions) vers quel direction nous allons. C'est donc la phase de sélection.

\subsubsection*{Génération}
Une fois le nœud sélectionné, nous utiliserons une des bornes de la dernière contrainte rencontrée pour générer une première valeur d'entrée au programme satisfaisant à la limite une des bornes de la contrainte. 
Par exemple pour une contrainte où $x < 10$ où $x$ est un entier positif et paramètres d'entrées de la fonction, nous utiliserons $x = 9$. Si la contrainte exprimée par ce nœud, ne peut être affectée par une valeur d'entrée alors elle sera ignorée et nous passerons à la prochaine étape de sélection.
Nous avons fait le choix de sélectionner des valeurs bornant les contraintes de chemins dans un premier temps car ce sont souvent sur les valeurs bornés que les apprenants se trompent lors de la rédaction de leurs programmes (exemple: $n-1 < 10$ au lieu de $n < 10$).

\subsubsection*{Simulation}
Grâce à cette valeur d'entrée que nous avons généré, nous pouvons donc l'injecter dans le programme de l'apprenant et de la correction pour détecter des éventuelles divergences de comportement, c'est une exécution concolique. On suppose d'ailleurs que si le programme plante pour une valeur donnée alors cela est considéré immédiatement comme une divergence.
De plus c'est cette exécution concolique qui pourrait permettre de résoudre le problème de complexité des contraintes. Si elles deviennent trop importantes, elles pourraient être remplacées par des valeurs concrètes obtenus lors de l'exécution du programme et faciliter l'exécution des solveurs de contraintes.

\subsubsection*{Propagation}
Une divergence dans notre cas correspond à un succès et une correspondance de comportement à un échec. Nous décidons donc si une divergence est détectée, de remonter l'information  au nœud du chemin parcouru. Cette information, nous servira à classifier les instructions du langage menant à une divergence en supposant que ces instructions, pourront aussi mener à des divergences sur d'autres programmes. En parallèle de ça, nous mettons la valeur générée de côté pour en faire un éventuel jeu de test.\\

Une fois la simulation exécutée, nous retournons à l'étape de sélection et ainsi de suite jusqu'à ce que suffisamment de jeux de tests montrant une divergence ont été trouvés ou que le temps de calcul pour l'analyse qui devrait être fixé au préalable a été écoulé.\\

Une fois cette méthode posée, quelques questions se posent encore:
\begin{itemize}
\item si un nombre conséquent de jeux de test montrant une divergence de comportement sur ce programme ont été trouvés, comment filtrer ceux qui le seront sur d'autres programmes ?
\item comment transposer les données des nœuds des branches pour les appliquer sur d'autres arbres d'exécution d'autres programmes ?
\end{itemize}

\subsubsection*{Sélection des valeurs générées}
Dans le cas où de nombreuses valeurs seraient générées nous pouvons simplement décider de n'utiliser que les valeurs de tests qui détectent une divergence sur au moins $k$ programmes parmi $n$. L'avantage de notre méthode réside donc dans le nombre d'itérations de notre analyse sur tous les programmes pour y détecter des erreurs récurrentes. On peut imaginer que si tous les programmes éronnés implémentent chacun un type d'erreur différent, alors notre méthode sera caduque.

\subsubsection*{Correspondance de la classification}
Les contraintes de chemins d'un programme qui font l'objet d'une divergence auront été classifiés pour favoriser le parcours de ces chemins afin de détecter d'autres erreurs dans d'autres programmes.
Pourtant, il n'est pas dit que les autres programmes puissent correspondre en terme de contraintes puisque deux programmes peuvent implémenter la même fonctionnalité avec une structure interne complètement différente. 
Nous pensons tout de même qu'il existe quelques techniques pour valoriser les analyses précédentes. Nous pouvons dans un premier utiliser des techniques de renommage pour uniformiser les programmes. De plus, nous pouvons avec l'exécution symbolique associer un nœud avec le code associé et éventuellement y récupérer les instructions utilisées. Ces informations peuvent être précieuses puisqu'elles pourraient permettre d'identifier des \textit{pattern} récurrents dans l'ordre d'exécution des instructions qui sont à l'origine d'erreur.

\subsection{Problématiques restantes}
Nous avons présenté une démarche tirant parti des techniques du domaine du test avec l'exécution symbolique et concolique associé à une heuristique de recherche avec apprentissage avec le \textit{Monte Carlo Tree Search}. Cette présentation n'est qu'une ébauche, sa réalisation nécessite de maitriser des outils d'exécution symbolique qui sont souvent complexes et de pouvoir l'instrumenter pour tenir compte de notre cas d'utilisation, ce qui demande un certain temps. Outre les problématiques associées à l'évaluation de la capacité de cette méthode à répondre à notre besoin, nous avons d'autres problématiques qui se posent aussi et qui devront être complétées.\\

La première est la détermination de l'ordre dans lequel les programmes sont analysés. Comme nous tirons parti des techniques de classification, il est important que les premiers programmes puissent nous apporter une quantité d'informations suffisantes pour guider les prochaines analyses. Nous pensons qu'il est important de limiter la quantité de programmes à analyser et c'est pourquoi nous avons présenter une technique de filtrage mais il doit aussi être possible de déterminer l'ordre dans lequel les programmes seront analysés selon certains critères. Nous pouvons par exemple, analyser le premier programme de manière aléatoire et si celui ci ne permet pas de détecter d'erreur, alors nous pourrions sélectionner le prochain programme selon sa distance avec le programme courant. Par distance on entend le nombre de différence qu'un programme aurait dans sa structure par rapport à un autre, comme nous pourrions exprimer la distancer entre deux chaînes de caractères. Ceci n'est bien sur qu'une brève proposition mais il doit exister des techniques plus efficaces.\\

Le nombre de simulation est déterminant pour s'assurer de la qualité du parcours du MCTS. Dans notre cas, nous ne simulons pas réellement puisque nous exécutons le programme avec des valeurs générés (conjointement avec le programme de la correction). Il est très probable que cette simulation soit problématique à cause du temps d'exécution possible des programmes. Bien qu'on soit dans un cadre d'apprentissage, il est possible que des programmes mal écrit soit particulièrement lent mais correct. Il faut donc trouver des solutions pour diminuer cette complexité, l'idéal serait bien sur de ne pas exécuter le programme mais de le simuler complètement.

%----------------------------------------------------------------------------------------
%	CHAPITRE 3
%----------------------------------------------------------------------------------------
\chapter*{Conclusion}
\addchaptertocentry{Conclusion}



%----------------------------------------------------------------------------------------
%	THESIS CONTENT - APPENDICES
%----------------------------------------------------------------------------------------

% \appendix % Cue to tell LaTeX that the following "chapters" are Appendices

% Include the appendices of the thesis as separate files from the Appendices folder
% Uncomment the lines as you write the Appendices

% \include{Appendices/AppendixA}
% \include{Appendices/AppendixB}
% \include{Appendices/AppendixC}

%----------------------------------------------------------------------------------------
%	BIBLIOGRAPHY
%----------------------------------------------------------------------------------------

\printbibliography[heading=bibintoc]

%----------------------------------------------------------------------------------------

\end{document}  
